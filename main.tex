\documentclass{article}
\usepackage[utf8]{inputenc}

\title{The role of species abundance on epiphyte
host interactions}
\author{Sara Mortara, Marilia Gaiarsa, Daniela Talora & André Amorim}
\date{September 2018}

\begin{document}

\maketitle

\begin{abstract}
    
\end{abstract}

\textbf{Key words:} abundance; ferns; metacommunity; network; Ombrophilous Forest; tree ferns


\section*{Introduction}

%Anotações gerais:

%Epiphyte-host interactions are usually described as comensalists networks and species are usually generalists in terms of their preferences for tree hosts.

%However, in order to understand preference one need to know the abundance distribution of both components of the network. For instance, a nested network is likely to emerge if both components of the network exhibit a skewed abundance distribution such as log-normal.

%For ferns, there is the hypothesis that fern epiphytes exhibit preferences for tree ferns as hosts given the favorable habitat for establishment of both generations of fern life cycle (Moran et al. 2003).

P1

How abundance affect network structure
abundance: Suweis + Fontaine + Vasquez & Aizen

P2

Epiphyte-host networks and interactions of epiphyte ferns with their host trees

P3

Goals and hypotheses

Based on empirical patterns of SADs of trees and epiphytes we aimed to understand how abundance of epiphytes and their hosts affects the structure of the epiphyte-host network. We addressed the following questions: (i) Does species abundance drive nestedness pattern? (ii) Does epiphyte species prefer tree ferns as hosts? (iii) Does epiphyte species prefer bigger trees? (iv) Are epiphyte species interacting positively among each other? First, we expect a left-skewed SAD distribution for both epiphytes and hosts would result in nested networks. Second, if epiphyte species have a preference for tree ferns, we would observe higher richness in tree ferns than angiosperm trees and a modular structure in the epiphyte-host network. Third, if epiphyte species prefer bigger trees, we would find a positive relation between host tree dbh and epiphyte richness. Finally, if epiphyte species are positively interacting among each other, we would expect pairs of species to exhibit more positive co-occurrence patterns than expected by chance.

\section*{Methods}

\subsection*{Study area}

\subsection*{Community sampling}

(i) Does epiphyte species prefer tree ferns as hosts?

We used individual-based rarefaction analysis to standardize richness values for epiphyte communities on angiosperm trees and tree ferns using vegan package.

(ii) Are there more species in bigger trees?

(iii) Are epiphyte species interacting positively among each other?

We applied the probabilistic model of species co-occurrence (Veech 2013) to epiphyte communities on host trees.

\section{Results}

Fig. 1
Table 1
	 	 	
We found 41 (N=) epiphyte species (including hemiepiphytes) occurring in 128 (N=357) tree species. Tree community in the study area comprises XX individuals from XX species from which only XX harbored species of epiphyte ferns. Species abundance distribution for both epiphytes and host trees showed a long tail distribution and followed a log-series model (Fig. 1). 



(i) Does epiphyte species prefer tree ferns as hosts?

Epiphyte richness on angiosperm trees and fern trees did not differ (Fig. 2). Even though angiosperm hosts harbor more total richness (S=41) than fern trees (S=20), individual-based rarefaction shows that for the same number of individuals richness is the same for both types of hosts.
(ii) Are there more species in bigger trees?
We did not find a positive relation of species richness and dbh for the overall community (Fig. 3). However, for epiphyte species on fern trees we found that epiphyte richness increases with host dbh.

(iii) Are epiphyte species interacting positively among each other?

We observed 41 epiphyte species distributed on 357 host individuals. Of 820 epiphyte-epiphyte pair combinations, we only analyzed 20\% (N=162) because 658 pairs had expected co occurrence < 1. We observed 44\% of non-random co-occurrences, being all of then negative. It is worth noting that the most abundant species in our data set (Asplenium auriculatum) interacts negatively with all other species.

\section*{Discussion}

\textbf{ACKNOWLEDGMENTS}

We thank Diogo Rocha for providing the raw data on tree community. We thank Vitor Becker, Clemira Oliveira and all Serra Bonita Private Reserve staff for support during data collection. We are pleased to CEPEC Herbarium for providing the infrastructure for the development of this work and to Fernando Mattos for helping with identification of fern species.


\textbf{Data Availability:} The data used in this study will be archived at the Dryad Digital Repository when accepted for publication.

\textbf{Author's contribution:} 

\section*{Literature cited}


\end{document}
